\begin{document}
Des del departament de GREIA de l'Universitat de Lleida, s'està treballant 
en una eina d'anàlisis de discusions que es donen en xarxes socials com Twitter, Reddit... Aquest treball consisteix en desenvolupar un modul més 
d'aquest projecte, una eina capaç de constriur un arbre conversacional 
a partir de la discusió que es forma en els comentaris d'un vídeo de Youtube.\\\\
%
Per tal de realitzar-ho, en primera instancia, s'haurà d'aconseguir
informació i context sobre el contingut del vídeo per a formar el node 
arrel de l'arbre i seguidament dels comentaris per tal d'anar formant 
les branques de conversa. Posteriorment, un cop construit l'arbre, aquest
s'haura de serialitzar en un fitxer XML amb un format determinat per a que
l'eina d'analisi de discusions ho pugui entendre.
Aquest format consisteix en una component on es defineixen per una banda, 
els nodes del graf arbre i per altra, les arestes que els uneixen.\\\\
%
Finalment, s'integrarà aquesta eina a una aplicació web desenvolupada 
per a fer l'interacció entre aquesta i l'analitzador de discusions
de forma més interactiva i còmoda.\\\\
%
S'ha decidit estructurat el projecte en tres fases: 
\begin{itemize}
\item Una primera fase de recollida de dades: context del video, likes, comentaris...
\item Una segona de transformació de les dades: construcció de l'arbre i parsejat a XML.
\item I per últim, una tercerpart d'integració amb l'aplicació web i l'analitzador de discusions.
\end{itemize}
\end{document}