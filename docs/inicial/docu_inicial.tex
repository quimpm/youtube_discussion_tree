% !TeX spellcheck = ca
\documentclass{article}
\usepackage[utf8]{inputenc}
\usepackage{graphicx}
\usepackage{hyperref}
\usepackage{amsmath}
\usepackage{ amssymb }
\usepackage{tikz}
\usepackage{float}
\usepackage[simplified]{pgf-umlcd}
\usepackage{subfiles}
\usepackage{listings}
\usepackage[lighttt]{lmodern}
\usepackage{color}
\usepackage[nottoc,numbib]{tocbibind}

\graphicspath{ {img/} }

\usetikzlibrary{positioning,fit,calc,arrows.meta, shapes, snakes}
%Tot això hauria d'anar en un pkg, però no sé com és fa
\newcommand*{\assignatura}[1]{\gdef\1assignatura{#1}}
\newcommand*{\grup}[1]{\gdef\3grup{#1}}
\newcommand*{\professorat}[1]{\gdef\4professorat{#1}}
\renewcommand{\title}[1]{\gdef\5title{#1}}
\renewcommand{\author}[1]{\gdef\6author{#1}}
\renewcommand{\date}[1]{\gdef\7date{#1}}
\renewcommand{\baselinestretch}{1.5}
\renewcommand{\maketitle}{ %fa el maketitle de nou
	\begin{titlepage}
		\raggedright{UNIVERSITAT DE LLEIDA \\
			Escola Politècnica Superior \\
			Grau en Enginyeria Informàtica\\
			\1assignatura\\}
		\vspace{5cm}
		\centering\huge{\5title \\}
		\vspace{3cm}
		\large{\6author} \\
		\normalsize{\3grup}
		\vfill
		Director : \4professorat \\
		Data : \7date
\end{titlepage}}
%Emplenar a partir d'aquí per a fer el títol : no se com es fa el package
%S'han de renombrar totes, inclús date, si un camp es deixa en blanc no apareix


	%Style of nodes. Si poses aquí un estil es pot reutilitzar més facilment
	\makeatletter
	\tikzset{
		database/.style={
			path picture={
				\draw (0, 1.5*\database@segmentheight) circle [x radius=\database@radius,y radius=\database@aspectratio*\database@radius];
				\draw (-\database@radius, 0.5*\database@segmentheight) arc [start angle=180,end angle=360,x radius=\database@radius, y radius=\database@aspectratio*\database@radius];
				\draw (-\database@radius,-0.5*\database@segmentheight) arc [start angle=180,end angle=360,x radius=\database@radius, y radius=\database@aspectratio*\database@radius];
				\draw (-\database@radius,1.5*\database@segmentheight) -- ++(0,-3*\database@segmentheight) arc [start angle=180,end angle=360,x radius=\database@radius, y radius=\database@aspectratio*\database@radius] -- ++(0,3*\database@segmentheight);
			},
			minimum width=2*\database@radius + \pgflinewidth,
			minimum height=3*\database@segmentheight + 2*\database@aspectratio*\database@radius + \pgflinewidth,
		},
		database segment height/.store in=\database@segmentheight,
		database radius/.store in=\database@radius,
		database aspect ratio/.store in=\database@aspectratio,
		database segment height=0.1cm,
		database radius=0.25cm,
		database aspect ratio=0.35,
	}
	\makeatother
\title{Document Inicial}
\author{Joaquim Picó Mora}
\date{XXXX-XX-XX}
\assignatura{Treball de fi de grau}
\professorat{R.Bejar}
\grup{49383707Q}

\renewcommand{\refname}{Bibliografia}

%Comença el document
\begin{document}
	\tableofcontents
	\newpage
	\pagenumbering{arabic}
	\section{Introducció}
	Les xarxes socials són un concepte originat en la comunicació.
	Es refereix al conjunt de grups, comunitats i organitzacions vinculats
	els uns als altres a través de relacions socials. A internet,
	existeixen moltes plataformes que implementen aquest tipus de
	comunicació mitjançant xarxes socials.
	En aquestes plataformes la comunicació succeeix quan usuaris
	publiquen contingut per a altres usuaris,
	de forma que els permet interaccionar entre ells mitjançant
	comentaris, likes, dislikes, reposts...
	Donat que, aquestes xarxes socials tenen un nombre exorbitant
	d'usuaris que interactuen cada dia amb altres usuaris, s'han convertit
	en una font molt rica de dades i en espais molt interessants d'investigar.\\\\
	%
	Des del departament de GREIA, s'està treballant amb una eina
	d'anàlisi de discussions que es donen en algunes de les principals
	Xarxes Socials que podem trobar a internet, entre elles Twitter, Reddit...
	%
	Cada Xarxa social, té una forma diferent de tractar les seves dades
	i formes diferents d'accedir-les, fent així que es requereixi
	d'una implementació particular per a cada una d'elles que es vulgui
	integrar a l'eina.
	\section{Youtube}
	Youtube és una plataforma web d'origen estatunidenc dedicada a què els
	seus usuaris hi compateixin vídeos. Va ser creada per 3 antics empleats de Paypal
	el febrer de 2005 i l'octubre de 2006 va ser adquirida per Google per 1650
	milions de dòlars. Actualment, és el lloc web més utilitzat en la seva categoria.\\\\
	%
	Quant a números, Youtube compta amb la increïble
	xifra de més de dos bilions d'usuaris repartits en 100 països
	que poden gaudir de contingut en 80 llengües, tot aglomerant diàriament
	un total de més d'un bilió d'hores de visualització de contingut
	de la plataforma.\\\\
	%
	Està compost en dues parts, una primera enfocada a consumidor, la qual consta
	d'una pàgina principal amb recomanacions personalitzades i un buscador
	des d'on buscar mitjançant paraules claus els vídeos que es desitja veure.
	I una segona enfocada als creadors de contingut on tindran un espai per
	gestionar el seu canal i els seus vídeos. Ens referim a canal com a espai on
	hi residiran els vídeos penjats per un usuari. Qualsevol usuari de Youtube
	pot ser tant consumidor com creador de contingut,
	de fet, tots els usuaris de youtube, ni que només en siguin consumidors,
	tenen un canal associat.\\\\
	%
	Un cop s'hi penja un vídeo de forma pública, altres usuaris hi poden accedir
	per tal de veure'l i a més poder interaccionar de diverses formes amb l'usuari
	que l'ha publicat. Poden donar-li un like si els ha agradat o un dislike si
	no els ha agradat, i poden escriure un comentari. Aquest comentari, també
	és accessible per la resta d'usuaris, i també hi poden interactuar de la mateixa
	forma, donant like, dislike o responent a aquest comentari.
	Aquest fet fa de Youtube un espai on hi tenen lloc discussions molt interessants
	sobre infinitud de temes, que es desenvolupen en els comentaris
	partint del contingut d'un vídeo.
	\section{Objectius del Treball}
	En aquest treball, es realitzarà la integració de la Xarxa Social Youtube
	a l'eina d'anàlisi de discussions desenvolupada per grup de recerca GREIA.
	L'objectiu principal constarà en realitzar una implementació d'una eina
	capaç de construir un arbre de discussió a partir d'un vídeo de Youtube i els
	comentaris que hi té associats.
	\section{Índex de la Memòria}
	\renewcommand{\labelenumii}{\Roman{enumii}}	
	\begin{enumerate}
		\item Introducció
		\item Descripció del Problema
		\item Youtube
		\item Recollida de Dades
		\begin{enumerate}
			\item Descripció
			\item Tecnologies
			\item Anàlisis
			\item Implementació
			\item Tests
		\end{enumerate}
		\item Construcció de l'arbre de discussió
		\begin{enumerate}
			\item Descripció
			\item Tecnologies
			\item Anàlisis
			\item Implementació
			\item Tests
		\end{enumerate}
		\item Serialització de l'arbre de discussió
		\begin{enumerate}
			\item Descripció
			\item Tecnologies
			\item Anàlisis
			\item Implementació
			\item Tests
		\end{enumerate}
		\item Integració
		\begin{enumerate}
			\item Descripció
			\item Eina Web Interactiva
			\item Eina d'anàlisi de discussions
		\end{enumerate}
		\item Conclusions
		\item Bibliografía
	\end{enumerate}
	\section{Tasques a realitzar}
	\begin{itemize}
		\item Anàlisi de l'API de Youtube
		\item Investigar sobre eines per a realitzar transcripcions de vídeos
		\item Recollida de Dades de Youtube
		\item Construcció de l'arbre de discussió
		\item Serialització de l'arbre de discussió en el format XML descrit per l'eina d'anàlisi.
		\item Realitzar l'anàlisi de sentiment
		\item Integració amb l'Eina Web Interactiva 
		\item Documentació de tot el procés
	\end{itemize}
	\section{Planificació temporal de les tasques a realitzar}	
	\begin{itemize}
		\item \textbf{Setmana 22 Març}
		\begin{enumerate}
			\item Investigar forma de recollir dades de Youtube
		\end{enumerate}
		\item \textbf{Setmana 29 Març}
		\begin{enumerate}
			\item  Investigar forma de recollir dades de Youtube
			\item Documentar procés
		\end{enumerate}
		\item \textbf{Setmana 5 Abril}
		\begin{enumerate}
			\item Planificar i disenyar la implementació
		\end{enumerate}
		\item \textbf{Setmana 12 Abril}
		\begin{enumerate}
			\item Planificar i disenyar la implementació
		\end{enumerate}
		\item \textbf{Setmana 19 Abril}
		\begin{enumerate}
			\item Implementació de la part de recullida de dades
		\end{enumerate}
		\item \textbf{Setmana 26 Abril}
		\begin{enumerate}
			\item Testos Recullida de dades
			\item Documentar la part de recullida de dades
		\end{enumerate}
		\item \textbf{Setmana 3 Maig}
		\begin{enumerate}
			\item Investigar i pensar amb l'algoritme de generació de l'Arbre
		\end{enumerate}
		\item \textbf{Setmana 10 Maig}
		\begin{enumerate}
			\item Implementació de l'algoritme de generació de l'arbre
		\end{enumerate}
		\item \textbf{Setmana 17 Maig}
		\begin{enumerate}
			\item Seguir amb la implementació  del  algoritme de generació de l'arbre
		\end{enumerate}
		\item \textbf{Setmana 24 Maig}
		\begin{enumerate}
			\item Testos del algoritme de generació de l'arbre
			\item Documentació de la fase de generació de l'arbre
		\end{enumerate}
		\item \textbf{Setmana 31 Maig}
		\begin{enumerate}
			\item Implementar l'algoritme de serialització de l'arbre
			\item Testos de l'algoritme de serialització
		\end{enumerate}
		\item \textbf{Setmana 07 Juny}
		\begin{enumerate}
			\item Documentació de la fase de serialització
			\item Modificar canvis que hi hagin pogut haver en el diseny i documentació de l'implementació en general
		\end{enumerate}
		\item \textbf{Setmana 14 Juny}
		\begin{enumerate}
			\item Integració amb l'eina web
		\end{enumerate}
	\end{itemize}
\end{document}
